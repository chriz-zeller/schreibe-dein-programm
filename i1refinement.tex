% Diese Datei ist Teil des Buchs "Schreibe Dein Programm!"
% Das Buch ist lizensiert unter der Creative-Commons-Lizenz
% "Namensnennung - Weitergabe unter gleichen Bedingungen 4.0 International (CC BY-SA 4.0)"
% https://creativecommons.org/licenses/by-sa/4.0/deed.de

\chapter{Schrittweise Verfeinerung}
\label{cha:iterative-refinement}

\index{schrittweise Verfeinerung}\index{Verfeinerung!schrittweise} In
der Praxis gelingt es nur selten, Programme erst vollständig zu planen
und dann umzusetzen: Manchmal ist die Aufgabenstellung bei
Entwicklungsbeginn noch nicht vollständig bekannt.  Manchmal ist die
Aufgabenstellung zwar bekannt aber unübersichtlich. Manchmal ändert
sich die Aufgabenstellung nachträglich.  Deshalb muss die Entwicklung
oft beginnen, bevor die gesamte Planung steht.  Das Programm wird dann
bis zu einem bestimmten Grad entwickelt und dann \textit{schrittweise
  verfeinert}, um die noch vorher unbekannten Aspekte der
Aufgabenstellung abzudecken.  Dabei ist es wichtig, zwischenzeitlich
gemachte Annahmen klar herauszustellen, fehlerhafte Programmteile auch
wieder bereitwillig zu löschen, und Schnittstellen gegen Änderungen zu
isolieren.  In diesem Kapitel werden verschiedene Beispiele und
Techniken für diese schrittweise Verfeinerung vorgestellt.

\section{Löschen in Suchbäumen}
\label{sec:search-tree-delete}

Das Löschen eines Elements in einem Suchbaum ist deutlich schwieriger
als das Einfügen.  Beim Suchbaum \texttt{s3} erscheint die Aufgabe
noch einfach.  Hier das Bild dazu:
%

TBD

% \begin{pspdf}
% \begin{center}
%   \pstree[levelsep=4ex,treesep=3em,nodesep=2pt]{\Tr{3}}
%   {
%     \Tdot
%     \pstree{\Tr{17}}{
%       \pstree{\Tr{5}}{\Tdot\Tdot}
%       \Tdot
%     }
%   }
% \end{center}
% \end{pspdf}
% 
Angenommen, die 3 soll gelöscht werden. In diesem Fall kann der
Suchbaum durch dessen rechten Teilbaum ersetzt werden:
%

TBD

% \begin{pspdf}
% \begin{center}
%     \pstree[levelsep=4ex,treesep=3em,nodesep=2pt]{\Tr{17}}{
%       \pstree{\Tr{5}}{\Tdot\Tdot}
%       \Tdot
%       }
% \end{center}
% \end{pspdf}
%
Angenommen, die 17 soll gelöscht werden: Dann kann der Teilbaum mit 17 an
der Spitze durch seinen eigenen linken Teilbaum ersetzt werden:
%
TBD

% \begin{pspdf}
% \begin{center}
%   \pstree[levelsep=4ex,treesep=3em,nodesep=2pt]{\Tr{3}}
%   {
%     \Tdot
%     \pstree{\Tr{5}}{\Tdot\Tdot}
%   }
% \end{center}
% \end{pspdf}
%
Ähnlich kann der Teilbaum mit der 5 an der Spitze durch den leeren
Baum ersetzt werden:
%

TBD

% \begin{pspdf}
% \begin{center}
%   \pstree[levelsep=4ex,treesep=3em,nodesep=2pt]{\Tr{3}}
%   {
%     \Tdot
%     \pstree{\Tr{17}}{
%       \Tdot
%       \Tdot
%     }
%   }
% \end{center}
% \end{pspdf}
%
Wird der Suchbaum nur minimal erweitert, wird die Sache schon schwieriger:
%

TBD

% \begin{pspdf}
% \begin{center}
%   \pstree[levelsep=4ex,treesep=3em,nodesep=2pt]{\Tr{3}}
%   {
%     \Tdot
%     \pstree{\Tr{17}}{
%       \pstree{\Tr{5}}{\Tdot\Tdot}
%       \pstree{\Tr{20}}{\Tdot\Tdot}
%     }
%   }
% \end{center}
% \end{pspdf}
%
Wie könnte hier die 17 gelöscht werden? Es ist nicht möglich, den
Teilbaum mit 17 an der Spitze einfach durch seinen linken oder rechten
Teilbaum zu ersetzen, weil dann der jeweils andere Teilbaum unter den
Tisch fiele.  Es ist also notwendig, die Baumstruktur zu
reorganisieren.  Mit dieser Beoachtung ist klar, daß die Aufgabe nicht
einfach dadurch zu lösen ist, daß die Konstruktionsanleitung einmal
befolgt wird: Stattdessen empfiehlt es sich, etwas nachzudenken und
zumindest eine grobe Lösungsstrategie zu entwickeln, bevor mit dem
Programmieren begonnen wird.

Eine Möglichkeit für eine Lösung wäre, den kompletten Suchbaum neu zu
bilden: Dazu könnte die Lösch-Funktion alle Elemente in einer Liste
aufsammeln, und dann mit \texttt{search-tree-insert} sukzessive aus
allen Elementen bis auf das zu löschende einen neuen Suchbaum
konstruieren.  (Siehe Aufgabe~\ref{ex:naive-delete}.)  Dies wäre
allerdings eine zeitaufwendige Methode, die überproportional länger
dauert, je größer der Suchbaum ist.

Besser wäre also, wenn beim Löschen soviel wie möglich des restlichen
Baums intakt bleibt: das soll für die Überlegungen erst einmal die
Maxime sein.  Hypothetisch wäre es möglich, daß dieser Ansatz nicht
zum Ziel führt und diese Annahme wieder rückgängig gemacht werden
muss.  Darum ist es sinnvoll, die Annahme deutlich sichtbar
zu dokumentieren:

\begin{quote}
  \textit{Annahme:} Es ist möglich, ein Element zu löschen, indem nur
  der Teilbaum verändert wird, an dessen Wurzel das Element sitzt; der
  Rest soll gleichbleiben.
\end{quote}

Etwas Nachdenken ergibt, daß der oben beschriebene
Ansatz, nur den Teilbaum zu ersetzen, an dessen Spitze das zu
löschende Element steht und den restlichen Baum unverändert zu lassen,
auch hier funktioniert.  Es gibt zwei Möglichkeiten, dies zu tun:
%

TBD

% \begin{pspdf}
% \begin{center}
%   \pstree[levelsep=4ex,treesep=3em,nodesep=2pt]{\Tr{3}}
%   {
%     \Tdot
%     \pstree{\Tr{5}}{
%       \Tdot
%       \pstree{\Tr{20}}{\Tdot\Tdot}
%     }
%   }
%   \qquad
%   \pstree[levelsep=4ex,treesep=3em,nodesep=2pt]{\Tr{3}}
%   {
%     \Tdot
%     \pstree{\Tr{20}}{
%       \pstree{\Tr{5}}{\Tdot\Tdot}
%       \Tdot
%     }
%   }
% \end{center}
% \end{pspdf}
% 
Mit anderen Worten, es wird ein Element von weiter unten
"<hochgezogen">.  Es bleibt die Frage, wie die Funktion das Element
auswählen könnte, das hochgezogen werden muss.  Offensichtlich kann
entweder ein Element aus dem linken oder aus dem rechten Teilbaum
ausgewählt werden.  Ideal wäre, wenn der jeweils andere Teilbaum
intakt bliebe.  Es muss also ein Element so hochgezogen werden, daß die
Suchbaumeigenschaft nicht verletzt wird.  Angenommen, das Element soll
aus dem linken Teilbaum genommen werden: Dann muss es größer sein als
alle anderen Elemente des linken Teilbaums.  (Es muss auch kleiner als
alle Elemente des rechten Teilbaums sein, aber dies ist automatisch
gegeben, weil die Suchbaumeigenschaft im ursprünglichen Baum bereits
gilt.)  Das hochzuziehende Element des linken Teilbaums wäre  also das
\emph{Maximum} aller Elemente des linken Teilbaums.  (Umgekehrt wäre
es genauso möglich, das Minimum der Elemente des rechten Teilbaums
auszuwählen: reine Geschmacksfrage.)

Zu diesem Zeitpunkt stellen sich möglicherweise bereits Kopfschmerzen
beim Programmierer ein: Wie wird das hochzuziehende Element gefunden?
Wie funktioniert der Hochziehprozeß selbst?  Daß diese Aufgaben lösbar
sind, ist klar, aber zusammen mit den anderen zu lösenden Teilaufgaben
vom Anfang dieses Abschnitts ist es vielleicht etwas viel auf einmal.

Darum ist es sinnvoll, erst einmal die schon klaren Teile der
Aufgabenlösung umzusetzen und die noch unklaren Teile per Wunschdenken
auf später zu verschieben.  Ein Teil der Strategie ist bereits
vollständig erkennbar: Das Löschen funktioniert, indem der Teilbaum,
an dessen Wurzel das zu löschende Element klebt, durch einen anderen
Baum ersetzt wird; der Rest des Baums ist intakt.  Es gibt also
bereits zwei erkennbare Teilaufgaben:
%
\begin{enumerate}
\item Den zu ersetzenden Teilbaum finden.
\item Aus einem Baum mit einem zu löschenden Element an der Spitze
  einen anderen Baum machen, der das Element nicht mehr enthält.
\end{enumerate}
%
Insbesondere müsste es leicht möglich sein, die zweite Teilaufgabe
soweit zu lösen, daß die Lösch-Funktion zu zumindest für \texttt{s3}
funktioniert:
%
\begin{verbatim}
; Element aus Suchbaum löschen
(: search-tree-delete (%a (search-tree %a) -> (search-tree %a)))

(check-expect (search-tree-tree (search-tree-delete 3 s3))
              (make-node 17
                         (make-node 5 the-empty-tree the-empty-tree)
                         the-empty-tree))
(check-expect (search-tree-tree (search-tree-delete 5 s3))
              (make-node 3
                         the-empty-tree
                         (make-node 17 the-empty-tree the-empty-tree)))
(check-expect (search-tree-tree (search-tree-delete 17 s3))
              (make-node 3 the-empty-tree (make-node 5 the-empty-tree the-empty-tree)))

(check-expect (search-tree-tree (search-tree-delete 28 s3))
              (search-tree-tree s3))
\end{verbatim}
%
Das Gerüst ist das gleiche, das schon für \texttt{search-tree-member?}
und \texttt{search-tree-insert} entwickelt wurde.  Die zwei
Hilfsfunktionen werden mit \texttt{letrec} gebunden:
%
\begin{verbatim}
(define search-tree-delete
  (lambda (e s)
    (let ((=proc (search-tree-label-equal-proc s))
          (<proc (search-tree-label-less-than-proc s)))
      (letrec
          ; Element finden und löschen
          ; (: delete ((tree %a) -> (tree %a))
          ((delete
            (lambda (t)
              ...))
           ; Element an Wurzel löschen
           ; (: delete-top ((node %a (tree %a) (tree %a)) -> (tree %a))
           (delete-top
            (lambda (n)
              ...)))
        (make-search-tree
         =proc <proc
         (delete (search-tree-tree s)))))))
\end{verbatim}
%
Die Hilfsfunktion \texttt{delete} folgt dabei dem gleichen Muster wie
die Hilfsfunktionen in \texttt{search-tree-member?} und
\texttt{search-tree-insert}: Die rekursiven Aufrufe folgen der
Struktur des Baums gemäß der Suchbaumeigenschaft.  Wenn des gesuchte
Element gefunden ist, kann die Hilfsfunktion \texttt{delete-top} den
Rest der Arbeit übernehmen.  Bei \texttt{delete-top} sind bereits zwei
Fälle klar: Wenn ein Teilbaum leer ist, so kann der Baum durch den
jeweils anderen Teilbaum ersetzt werden:
%
\begin{verbatim}
(define search-tree-delete
  (lambda (e s)
    (let ((=proc (search-tree-label-equal-proc s))
          (<proc (search-tree-label-less-than-proc s)))
      (letrec
          ; Element finden und löschen
          ; (: delete ((tree %a) -> (tree %a))
          ((delete
            (lambda (t)
              (cond
                ((empty-tree? t) t)
                ((node? t)
                 (cond
                   ((=proc e (node-label t))
                    (delete-top t))
                   ((<proc e (node-label t))
                    (make-node (node-label t)
                               (delete (node-left-branch t))
                               (node-right-branch t)))
                   (else
                    (make-node (node-label t)
                               (node-left-branch t)
                               (delete (node-right-branch t)))))))))
           ; Element an Wurzel löschen
           ; (: delete-top ((node %a (tree %a) (tree %a)) -> (tree %a))
           (delete-top
            (lambda (n)
              (cond
                ((empty-tree? (node-left-branch n))
                 (node-right-branch n))
                ((empty-tree? (node-right-branch n))
                 (node-left-branch n))
                (else ...)))))
        (make-search-tree
         =proc <proc
         (delete (search-tree-tree s)))))))
\end{verbatim}
%
Die Testfälle mit \texttt{s3} funktionieren damit schon einmal.  Es
ist durchaus sinnvoll, bewußt die einfachen Fälle zuerst zu
programmieren und auch zu testen.  Gibt es später Probleme bei den
komplizierteren Fällen, so sind diese einfacher zu lokalisieren.

Der folgende Testfall allerdings beschwert sich noch über den
fehlenden Code im \texttt{else}-Zweig:
%
\begin{verbatim}
(define s4
  (make-search-tree
   = <
   (make-node 17
              (make-node 5 the-empty-tree the-empty-tree)
              (make-node 20 the-empty-tree the-empty-tree))))

(check-expect (search-tree-tree (search-tree-delete 17 s4))
              (make-node 5 the-empty-tree
                         (make-node 20 the-empty-tree the-empty-tree)))
\end{verbatim}
%
Für die Ellipse muss nun also das maximale Element des linken Teilbaums
hochgezogen werden; der rechte Teilbaum bleibt unangetastet.  Das
könnte etwa so aussehen:
%
\begin{verbatim}
(make-node (find-maximum (node-left-branch n))
           ...
           (node-right-branch n))
\end{verbatim}
%
Per Wunschdenken wird hier eine Funktion \texttt{find-maximum}
vorausgesetzt:
%
\begin{verbatim}
; Maximum von Teilbaum finden
(: find-maximum ((tree %a) -> %a))
\end{verbatim}
%
Aber auch im obigen Ausdruck fehlt noch etwas, nämlich der linke
Teilbaum des neu konstruierten Baums.  Der soll alle Elemente des
ursprünglichen Teilbaums finden, \emph{außer dem Maximum}.  Wie kann
\texttt{delete-top} das Maximum loswerden?  Am einfachsten durch die
Verwendung der Lösch-Funktion, die gerade geschrieben wird.  Erster
Anlauf:
%
\begin{verbatim}
(make-node (find-maximum (node-left-branch n))
           (search-tree-delete
             (find-maximum (node-left-branch n))
             (node-left-branch n))
           (node-right-branch n))
\end{verbatim}
%
Hier taucht \texttt{(find-maximum (node-left-branch n))} mehrfach auf,
es sollte also durch ein \texttt{let} gebunden werden:
%
\begin{verbatim}
(let ((el (find-maximum (node-left-branch n))))
  (make-node el
             (search-tree-delete
               el
               (node-left-branch n))
             (node-right-branch n)))
\end{verbatim}
%
Leider ist der Code so nicht richtig, da \texttt{(node-left-branch n)}
kein \texttt{search-tree}-Wert ist, sondern nur ein
\texttt{tree}-Wert: es gibt also eine Vertragsverletzung.  Der Baum
\texttt{(node-left-branch n)} muss also noch in ein
\texttt{search-tree}-Record eingewickelt werden.  Außerdem kommt bei
\texttt{search-tree-delete} wieder ein \texttt{search-tree}-Wert
heraus, während \texttt{make-node} einen \texttt{tree}-Wert erwartet,
der hinterher wieder ausgewickelt werden muss:
%

\begin{verbatim}
(let ((el (find-maximum (node-left-branch n))))
  (make-node el
             (search-tree-tree
               (search-tree-delete
                 el
                 (make-search-tree =proc <proc
                                   (node-left-branch n))))
             (node-right-branch n)))
\end{verbatim}
%
Alternativ zum umständlichen Ein-
und Auswickeln kann auch die \texttt{delete}-Hilfsfunktion um einen
Parameter für das zu löschende Element erweitert werden:
%
\begin{verbatim}
(letrec
    ; Element finden und löschen
    ; (: delete (%a (tree %a) -> (tree %a)))
    ((delete
      (lambda (e t)
        (cond
          ((empty-tree? t) t)
          ((node? t)
           (cond
             ((=proc e (node-label t))
              (delete-top t))
             ((<proc e (node-label t))
              (make-node (node-label t)
                         (delete e (node-left-branch t))
                         (node-right-branch t)))
             (else
              (make-node (node-label t)
                         (node-left-branch t)
                         (delete e (node-right-branch t)))))))))
\end{verbatim}
%
Dann lautet der Code im \texttt{else}-Zweig:
%
\begin{verbatim}
(let ((el (find-maximum (node-left-branch n))))
  (make-node el
             (delete el (node-left-branch n))
             (node-right-branch n)))
\end{verbatim}
%
Fehlt nur noch die Hilfsfunktion \texttt{find-maximum}.  Beim
Ermitteln des Maximums hilft die Suchbaumeigenschaft: Es ist einfach
das Element, das am weitesten rechts im Baum steht.  Die Funktion ist
also ganz einfach zu schreiben.  Komplett sieht
\texttt{search-tree-delete} dann so aus:
%
\begin{verbatim}
(define search-tree-delete
  (lambda (e s)
    (let ((=proc (search-tree-label-equal-proc s))
          (<proc (search-tree-label-less-than-proc s)))
      (letrec
          ; Element finden und löschen
          ; (: delete (%a (tree %a) -> (tree %a))
          ((delete
            (lambda (e t)
              (cond
                ((empty-tree? t) t)
                ((node? t)
                 (cond
                   ((=proc e (node-label t))
                    (delete-top t))
                   ((<proc e (node-label t))
                    (make-node (node-label t)
                               (delete e (node-left-branch t))
                               (node-right-branch t)))
                   (else
                    (make-node (node-label t)
                               (node-left-branch t)
                               (delete e (node-right-branch t)))))))))
           ; Element an der Wurzel löschen
           ; (: delete-top ((node %a (tree %a) (tree %a)) -> (tree %a))
           (delete-top
            (lambda (n)
              (cond
                ((empty-tree? (node-left-branch n))
                 (node-right-branch n))
                ((empty-tree? (node-right-branch n))
                 (node-left-branch n))
                (else
                 (let ((el (find-maximum (node-left-branch n))))
                   (make-node el
                              (delete el (node-left-branch n))
                              (node-right-branch n)))))))
           ; Maximum von Teilbaum finden
           ; (: find-maximum ((tree %a) -> %a))
           (find-maximum
            (lambda (t)
              (cond
                ((empty-tree? t)
                 (violation "unerwarteter leerer Baum"))
                ((node? t)
                 (let ((right (node-right-branch t)))
                   (if (empty-tree? right)
                       (node-label t)
                       (find-maximum (node-right-branch t)))))))))
           
        (make-search-tree
         =proc <proc
         (delete e (search-tree-tree s)))))))
\end{verbatim}
%
Beim Testen fällt auf, daß der rekursive Aufruf von
\texttt{find-maximum} noch nicht abgedeckt ist.  (Daß der Aufruf von
\texttt{violation} nicht abgedeckt ist, ist erwünscht~-- es soll ja
nicht passieren.)  Es muss also noch ein Testfall konstruiert werden:
%
\begin{verbatim}
(define s4
  (make-search-tree
   = <
   (make-node 17
              (make-node 5 the-empty-tree the-empty-tree)
              (make-node 20 the-empty-tree the-empty-tree))))

(check-expect (search-tree-tree (search-tree-delete 17 s4))
              (make-node 5 the-empty-tree
                         (make-node 20 the-empty-tree the-empty-tree)))
\end{verbatim}
%

% \begin{mantra}
%   Kill your darlings.
% \end{mantra}

\section{Datenverfeinerung}

Bei "<realen"> Programmieraufgaben sind selten schon präzise
Datendefinitionen vorgegeben: Stattdessen liegen oft nur ein paar vage
Anforderungen an die Software vor, die sich unter Umständen auch noch
mit der Zeit ändern.  In solchen Fällen ist es nicht möglich, wie bei
den meisten Übungsaufgaben dieses Buchs, die Datendefinition direkt
an der Aufgabenstellung abzulesen und alle zu anfallenden Probleme mit
einem Wurf zu lösen.

Das Beispiel dieses Abschnitts ist, die Verwaltungsdaten von Wohnungen
zu modellieren. Über eine Wohnung lässt sich viel sagen, von dem nur
manches je nach Kontext und Aufgabe wichtig ist.  Darum ist es
schwierig, im ersten Anlauf eine vollständige und präzise
Datendefinition zu entwerfen.  In diesem Abschnitt geht es darum, was
passiert, wenn sich die Anforderungen und damit die Datendefinitionen
ändern.  Angenommen, es geht zunächst einmal vage um die Adresse, die
Bewohner, die Zimmer und die Wohnfläche.  Immerhin sind daran schon
eine Daten- und eine Record-Definition ablesbar:
%
\begin{verbatim}
; Eine Wohnung besteht aus:
; - Adresse
; - Bewohner
; - Zimmern
(define-record-procedures apartment
  make-apartment apartment?
  (apartment-address apartment-resident apartment-rooms))

(: make-apartment (string string (list room) -> apartment))
\end{verbatim}
%
Hier ist allerdings bereits eine willkürliche Entscheidung gefallen:
Die Datendefinition enthält nicht direkt die Grundfläche der Wohnung.
Diese soll pro Zimmer angegeben werden.  Der Vertrag \texttt{room} für
Zimmer ist noch offen; da über die Zimmer außer der Grundfläche nichts
weiter bekannt ist, wäre eine Möglichkeit die folgende Definition:
%
\begin{verbatim}
(define-contract room rational)
\end{verbatim}
%
Diese naive Definition hat zwei Nachteile:
%
\begin{itemize}
\item Es gibt unmittelbar kein eindeutiges Prädikat für Zimmer.
\item Es ist klar, daß sich über ein Zimmer noch mehr Dinge sagen
  lassen als die Grundfläche: Früher oder später werden Zimmer zu
  zusammengesetzten Daten.
\end{itemize}
%
Aus diesen Gründen empfiehlt es sich, von vornherein von
zusammengesetzten Daten auszugehen:
%
\begin{verbatim}
; Ein Zimmer besteht aus:
; - Fläche
(define-record-procedures room
  make-room room?
  (room-area))

(: make-room (rational -> room))
\end{verbatim}
%
Hier sind zwei Wohnungen für Tests:
%
\begin{verbatim}
(define a1
  (make-apartment
   "Entenhausener Straße 15"
   "Sperber"
   (list (make-room 30) (make-room 50) (make-room 60))))
\end{verbatim}
%
%
\begin{verbatim}
(define a2
  (make-apartment
   "Froschgasse 17"
   "Crestani"
   (list (make-room 20) (make-room 40))))
\end{verbatim}
%
Die Gesamtfläche einer Wohnung lässt sich einfach durch Addieren der
Flächen der Zimmer berechnen:
%
\begin{verbatim}
; Fläche einer Wohnung berechnen
(: apartment-area (apartment -> number))            
(check-expect (apartment-area a1) 140)
(check-expect (apartment-area a2) 60)

(define apartment-area
  (lambda (a)
    (fold 0 +
          (map room-area (apartment-rooms a)))))
\end{verbatim}
%
An dieser Stelle wird eine positive Eigenschaft der bisherigen Daten-
und Funktiondefinitionen deutlich.  Angenommen, die Record-Definition
hätte die Grundfläche direkt einer Wohnung zugeordnet:
%
\begin{verbatim}
; Eine Wohnung besteht aus:
; - Adresse
; - Bewohner
, - Grundfläche
; - Zimmern
(define-record-procedures apartment
  make-apartment apartment?
  (apartment-address apartment-resident apartment-rooms apartment-area))

(: make-apartment (string string (list room) rational -> apartment))
\end{verbatim}
%
In diesem Fall würde der Selektor \texttt{apartment-area} sich nach
außen genauso verhalten wie die Funktion \texttt{apartment-area}~--
gleicher Vertrag. gleiche Funktion.  Die Tatsache, daß der
Record-Selektor eine reguläre Funktion ist, die sich von
"<selbstgeschriebenen"> Funktionen nach außen nicht unterscheidet,
isoliert Benutzer der Wohnungs-Funktionalität von den Unterschieden
zwischen den zwei Varianten.

Aus Sicht eines Wohnungsbesitzers interessieren möglicherweise noch
weitere Daten, zum Beispiel, ob ein Bewohner noch Untermieter hat.
Es ist sogar möglich, daß Untermieter ihrerseits selbst Untermieter
haben.  Dieser Idee ließe sich Rechnung tragen, indem eine Wohnung
statt aus Zimmern aus "<Abschnitten"> besteht, und jeder Abschnitt
entweder ein Zimmer oder ein untervermieter Wohnungsteil ist.  Die
Definition für Wohnungen müsste also folgendermaßen geändert werden:
%
\begin{verbatim}
; Eine Wohnung besteht aus:
; - Adresse
; - Bewohner
; - Abschnitten
(define-record-procedures apartment
  make-apartment apartment?
  (apartment-address apartment-resident apartment-sections))

(: make-apartment (string string (list section) -> apartment))
\end{verbatim}
%
Bei Abschnitten (\texttt{section}) handelt es sich klar um gemischte
Daten:
%
\begin{verbatim}
; Ein Abschnitt ist eins der folgenden:
; - ein Zimmer
; - ein untervermieter Teil der Wohnung
(define-contract section (mixed room sublet))
\end{verbatim}
%
Zimmer sind wie gehabt.   Untervermietete Wohnungsteile sind wie
Wohnungen, haben aber keine Adresse:
%
\begin{verbatim}
; Ein unvermieteter Teil des Wohnung besteht aus:
; - Bewohner
; - Abschnitten
(define-record-procedures sublet
  make-sublet sublet?
  (sublet-resident sublet-sections))

(: make-sublet (string (list section) -> sublet))
\end{verbatim}
%
Hier ist ein Beispiel für solche einen untervermieteten Teil sowie ein
Apartment, das es enthält:
%
\begin{verbatim}
(define s1
  (make-sublet "Taschenbier"
               (list (make-room 20) 
                     (make-sublet "Sams"
                                  (list (make-room 2)))
                     (make-room 10))))

(define a3
  (make-apartment
   "Flugentenschneise 17"
   "Rotkohl"
   (list (make-room 100)
         s1
         (make-room 30))))
\end{verbatim}
%
Es kann nun sein, daß die "<Wohnungs-Funktionalität"> schon in ein
größeres Programm eingebaut wurde, daß dessen Funktionen benutzt,
insbesondere den Selektor \texttt{apartment-rooms}.  Der ist im Zuge
der Erweiterung um Untermiete unter den Tisch gefallen.  Das Konzept
der Zimmer einer Wohnungs ist aber immer noch sinnvoll; entsprechend
ist es möglich, \texttt{apartment-rooms} wieder zur Verfügung zur
stellen, diesmal als "<normale"> Funktion mit einer Hilfsfunktion, um
die Zimmer eines Abschnitts aufzuzählen:
%
\begin{verbatim}
; Zimmer einer Wohnung aufzählen
(: apartment-rooms (apartment -> (list room)))

(check-expect (apartment-rooms a1)
              (list (make-room 30) (make-room 50) (make-room 60)))
(check-expect (apartment-rooms a3)
              (list (make-room 100)
                    (make-room 20)
                    (make-room 2)
                    (make-room 10)
                    (make-room 30)))

(define apartment-rooms
  (lambda (a)
    (fold empty append
          (map section-rooms (apartment-sections a)))))

; Zimmer eines Wohnungsabschnitts aufzählen
(: section-rooms (section -> (list room)))

(check-expect (section-rooms (make-room 30))
              (list (make-room 30)))
(check-expect (section-rooms s1)
              (list (make-room 20)
                    (make-room 2)
                    (make-room 10)))
              
(define section-rooms
  (lambda (s)
    (cond
      ((room? s) (list s))
      ((sublet? s)
       (fold empty append
             (map section-rooms (sublet-sections s)))))))
\end{verbatim}
%
Bei der Abspaltung von Hilfsfunktionen gibt es Ermessensspielraum:
Die Hilfsfunktion \texttt{section-rooms} ist notwendig, weil sie
rekursiv ist.

"<Von außen"> ist kein Unterschied zwischen dem alten Selektor
\texttt{apartment-rooms} und der neuen Funktion zur erkennen~-- die
Schnittstelle ist gleich geblieben: Dies ist ein wichtiger Beitrag zur
\textit{Modularität}\index{Modularität}, die es erlaubt, Änderungen an
der Wohnungs-Funktionalität vorzunehmen, ohne daß andere
Programmteile, welche die Funktionalität benutzen, verändert werden
müssen.

Zu den Programmteilen, die \texttt{apartment-rooms} verwenden,
gehört auch \texttt{apartment-area}, und tatsächlich funktioniert
\texttt{apartment-area} ohne Anpassungen wie vorher.  Außerdem
funktioniert es auch bei Wohnungen mit Untermietern, wie ein
zusätzlicher Testfall ermittelt:
%
\begin{verbatim}
(check-expect (apartment-area a3) 162)
\end{verbatim}
%
Mit den Erweiterungen an den Datendefinitionen sind allerdings auch
neue Funktionalitäten möglich.  Zum Beispiel könnte eine Funktion für
die Bewohner einer Wohnung präzise Adressen ausrechnen, etwa so:
%
\begin{verbatim}
; Adresse eines Bewohners berechnen
(: resident-address (string apartment -> string))

(check-expect (resident-address "Sperber" a1) "Sperber")

(check-expect (resident-address "Taschenbier" a3)
              "Taschenbier bei Rotkohl") 
(check-expect (resident-address "Sams" a3)
              "Sams bei Taschenbier bei Rotkohl")
\end{verbatim}
%
Die Funktion muss also gleichzeitig nach dem Bewohner suchen und die
Adreßangabe konstruieren.  Wenn der gesuchte Bewohner gerade der
"<Hauptbewohner"> der Wohnung ist, ist die Aufgabe einfach:
%
\begin{verbatim}
(define resident-address
  (lambda (r a)
    (if (string=? r (apartment-resident a))
        r
        ...)))
\end{verbatim}
%
Damit funktioniert bereits der erste Testfall. 

Die anderen Adreßangaben haben jetzt alle die Form \texttt{$x$ bei
$y$} wobei $y$ der Bewohner der Wohnung ist:
%
\begin{verbatim}
(define resident-address
  (lambda (r a)
    (if (string=? r (apartment-resident a))
        r
        (string-append ... " bei " (apartment-resident a)))))
\end{verbatim}
%
Der Präfix $x$ hängt vom Wohnabschnitt ab, in dem der gesuchte
Bewohner wohnt.  Per Wunschdenken sei eine Funktion vorausgesetzt, die
in den Abschnitten einer Wohnung nach dem Bewohner sucht und den
entsprechenden Präfix berechnet:
%
\begin{verbatim}
; Adreß-Präfix eines Abschnitts-Bewohners berechnen
(: sections-prefix (string (list section) -> string))
\end{verbatim}
%
Mit Hilfe dieser Funktion kann \texttt{resident-address} erweitert
werden:
%
\begin{verbatim}
(define resident-address
  (lambda (r a)
    (if (string=? r (apartment-resident a))
        r
        (string-append (sections-prefix r (apartment-sections a))
                       " bei "
                       (apartment-resident a)))))
\end{verbatim}
%
Nun zur Funktion \texttt{sections-prefix}.  Hier ein geeigneter
Testfall:
%
\begin{verbatim}
(check-expect (sections-prefix "Taschenbier" (apartment-sections a3))
              "Taschenbier")
\end{verbatim}
%
Gerüst und Schablone:
%
\begin{verbatim}
(define sections-prefix
  (lambda (r lis)
    (cond
      ((empty? lis) ...)
      ((cons? lis)
       ... (first lis) ...
       ... (sections-prefix r (rest lis)) ...))))
\end{verbatim}
%
Beim Ausfüllen der Schablone wird sofort klar, daß ein geeigneter
Rückgabewert für den \texttt{empty?}-Zweig noch fehlt: In diesem Fall
ist der Bewohner gar nicht gefunden worden.  Für diesen Umstand (der
in den Überlegungen bisher noch gar nicht auftauchte) muss
ein Extra-Wert eingeführt werden:
%
\begin{verbatim}
; Wert für nicht gefundenen Bewohner
(define-record-procedures not-found
  make-not-found not-found?
  ())

(define the-not-found (make-not-found))
\end{verbatim}
%
Entsprechend muss der Vertrag von \texttt{sections-prefix} erweitert
werden:
%
\begin{verbatim}
(: sections-prefix (string (list section) -> (mixed string not-found)))
\end{verbatim}
%
Da es sich bei \texttt{(first lis)} um gemischte Daten handelt, wird
eine Verzweigung fällig.  Wieder wird per Wunschdenken eine
Hilfsfunktion angenommen, diesmal \texttt{sublet-prefix}, die den
Adreß-Präfix für eine Untermiete ausrechnet, möglich:
%
\begin{verbatim}
(define sections-prefix
  (lambda (r lis)
    (cond
      ((empty? lis) the-not-found)
      ((cons? lis)
       (let ((f (first lis)))
         (cond
           ((room? f) (sections-prefix r (rest lis)))
           ((sublet? f)
            (let ((prefix (sublet-prefix r f)))
              (cond
                ((not-found? prefix)
                 (sections-prefix r (rest lis)))
                ((string? prefix)
                 prefix))))))))))
\end{verbatim}
%
Die Hilfsfunktion \texttt{sublet-prefix} soll sich folgendermaßen verhalten:
%
\begin{verbatim}
; Adreß-Präfix eines Untermieters bewohnen
(: sublet-prefix (string sublet -> (mixed string not-found)))

(check-expect (sublet-prefix "Taschenbier" s1) "Taschenbier")
(check-expect (sublet-prefix "Sams" s1) "Sams bei Taschenbier")
(check-expect (sublet-prefix "Merz" s1) the-not-found)
\end{verbatim}
%
Bei Untervermietungen ist die Arbeit dann erledigt, wenn der Bewohner
gerade der gesuchte ist:
%
\begin{verbatim}
(define sublet-prefix
  (lambda (r s)
    (if (string=? r (sublet-resident s))
        r
        ...)))
\end{verbatim}
%
Dies sieht ähnlich aus wie bei \texttt{apartment-prefix}; auch hier
funktioniert schon der erste Testfall.  Für die Alternative kann
\texttt{sections-prefix} benutzt werden:
%
\begin{verbatim}
(define sublet-prefix
  (lambda (r s)
    (if (string=? r (sublet-resident s))
        r
        (let ((prefix (sections-prefix r (sublet-sections s))))
          (cond
            ((not-found? prefix) the-not-found)
            ((string? prefix)
             (string-append prefix " bei " (sublet-resident s))))))))
\end{verbatim}
%
Damit sieht das Programm fast fertig aus. Allerdings ist der Änderung
im Vertrag von \texttt{sections-prefix} noch nicht überall Rechnung
getragen worden, namentlich nicht in \texttt{resident-address}:
Einerseits muss \texttt{resident-address} eine Fallunterscheidung für
den Rückgabewert von \texttt{sections-prefix} einführen; dies heißt
andererseits aber auch, daß \texttt{resident-address} selbst in die
Lage kommen kann, daß der gesuchte Bewohner nicht gefunden wird.
Damit werden neben der Erweiterung des Rumpfes der Funktion auch eine
Änderung des Vertrags und ein neuer Testfall fällig:
%
\begin{verbatim}
(: resident-address (string apartment -> (mixed string not-found)))
(check-expect (resident-address "Müller" a1) the-not-found)

(define resident-address
  (lambda (r a)
    (if (string=? r (apartment-resident a))
        r
        (let ((prefix (sections-prefix r (apartment-sections a))))
          (cond
            ((not-found? prefix)
             the-not-found)
            ((string? prefix)
             (string-append prefix
                            " bei "
                            (apartment-resident a))))))))
\end{verbatim}
%
Solche Änderungen an Verträgen, die im Rahmen der schrittweisen
Verfeinerung auftreten, sind wie Wellen im Teppichboden:  Sie müssen
durch die gesamte Ausdehnung des Programms weitergegeben werden, bis
sie schließlich am Rand angekommen sind.

Am Beispiel der Wohnungsverwaltung werden einige typische Einsichten
über die Software-Entwicklung deutlich:
%
\begin{itemize}
\item Einmal begonnene Software wird fast immer später erweitert und
  für Aufgaben eingesetzt, für die sie ursprünglich nicht gedacht war.
  Darum ist es sinnvoll, von vornherein die Möglichkeit späterer
  Erweiterungen zu berücksichtigen
\item Auch wenn scheinbar das gesamte Problem vor dem Beginn der
  Programm-Entwicklung bekannt ist, treten häufig noch beim
  Programmieren neue Einsichten zutage, die Anpassungen erfordern.
\item Änderungen an Verträgen~-- also Schnittstellen~-- ziehen häufig
  andere Änderungen in Funktionen nach sich, die diese benutzen.   Es
  ist wichtig, diese gewissenhaft durchzuführen.
\end{itemize}

In Mantra-Form sehen diese Einsichten so aus:
%
\begin{mantra}
  Programme wachsen.
\end{mantra}

\begin{mantra}
  Es ist nicht möglich, für alle Eventualitäten im voraus zu planen.
\end{mantra}

\begin{mantra}
  Änderungen an Schnittstellen müssen vollständig durch ein Programm
  propagiert werden.
\end{mantra}


\section*{Aufgaben}

\begin{aufgabe}\label{ex:naive-delete}
  Eine alternative Implementierung von \texttt{search-tree-delete}
  funktioniert folgendermaßen:
  %
  \begin{enumerate}
  \item Alle Elemente des Baums werden in einer Liste aufgezählt.
  \item Das zu löschende Element wird aus der Liste entfernt.
  \item Ein völlig neuer Suchbaum aus wird aus der bereinigten Liste
    konstruiert.
  \end{enumerate}
  %
  Realisieren Sie diese Alternative!  Wie unterscheidet sie sich in
  der Laufzeit von der im Text präsentierten Implementierung?
\end{aufgabe}

\begin{aufgabe}
  Zwischen den Record-Definitionen von \texttt{apartment} und
  \texttt{sublet} gibt es Gemeinsamkeiten.  Ist es möglich, von diesen
  Gemeinsamkeiten zu profitieren und den Code zu vereinfachen?
  Ändern Sie den Code entsprechend und beurteilen Sie, ob es sich
  gelohnt hat.
\end{aufgabe}

%%% Local Variables: 
%%% mode: latex
%%% TeX-master: "i1"
%%% End: 

